%% This document created by Scientific Word (R) Version 2.5
%% Starting shell: book
% \def\baselinestretch{1}


\documentclass[thmsa,a4paper,ukenglish]{report}
%%%%%%%%%%%%%%%%%%%%%%%%%%%%%%%%%%%%%%%%%%%%%%%%%%%%%%%%%%%%%%%%%%%%%%%%%%%%%%%%%%%%%%%%%%%%%%%%%%%%%%%%%%%%%%%%%%%%%%%%%%%%
\usepackage{sw20jrep}

%TCIDATA{TCIstyle=report/report.lat,jrep,report}

%TCIDATA{Created=Thu Sep 18 13:21:02 1997}
%TCIDATA{LastRevised=Mon Dec 01 10:53:15 1997}
%TCIDATA{Language=American English}

\input{tcilatex}
\parindent=0mm
\parskip=3mm
\textwidth=174mm
\textheight=254mm
\topmargin=-24mm
\oddsidemargin=-3mm
\evensidemargin=-3mm
\begin{document}

\author{H. R. Gunputh, R. Lengenfelder}
\title{Examples for Sarsim II}
\date{\today}

\chapter{Examples for Sarsim II}

\section{Introduction}

The discrete analysis of radar signal, which is nothing else than
electromagnetic waves sent in a specific pattern demands very powerful
computers. To undertake the task of writing a simulator is always a
compromise between accuracy and execution time. Many simulators have been
written to model the return of complex objects like aircraft or ship by
approximating them by a large number of surfaces. Although these simulations
(taking multiple reflections into account) give fairly reasonable
approximations for simple bodies, calculation times usually are measured in
hours.

A short description of the methods used in this case will be given:

The simulator used here models the reflective surface by means of very small
perfect `mirrors' called point targets. In theory everything could be
modelled by an extremely large number of these point targets and the
reflected signal would just be a linear combination of these. However for
most simulation purposes a small number of point targets will be used to get
a clear picture of what the effects of changing certain conditions entails.

A radar is considered to be the transmitter and receiver of the
electromagnetic pulses. Usually short bursts of energy are transmitted in
regular intervals at a certain carrier frequency. Sometimes the pulse is not
modulated (monochrome pulse), more often it is. A very common linear
frequency modulation is called `chirp'. The purpose of modulation is to
increase the pulse length (less peak power required) without sacrificing
resolution. After reception the signals are mixed to baseband, with other
words the carrier frequency component is removed. For the purpose of this
simulator, the output will always be at baseband, due to the simple reason
that sampling at typical carrier frequencies would require an enormous
number of samples. (In practice the signal would be mixed to some
intermediate frequency). The simulator therefore can only do simulations
which require baseband signals.

As the reflected pulse will be a replica of the transmitted pulse (although
scaled in amplitude and shifted in phase), the simulator does a rather
simple task. It calculates the distance and the relative radial velocity of
all given point targets and constructs a sampled return array by adding
`pulses' at calculated positions within the array. Power losses introduced
by range and antenna gains are taken into account. Effectively the simulator
takes the burden of calculating the geometry setup at any given time. For
very simple simulations, this might be trivial, however, for e.g.,
non-linear moving targets or special antenna gain patterns, a simulation is
the only practical option.

The following pages will describe simple scenarios which will be simulated.
The results will be verified by the applicable formulas.

\smallskip

\smallskip This next section will describe the use of Excel, Matlab and IDL
to view the simulation results saved in the ASCII files.

\subsection{\protect\smallskip Simulation results on Excel}

When the simulation is run, the raw return will display the actual waveform
pulse as shown in Figure \ref{raw.gif}. To see the individual pulses, edit
the range and azimuth values in the window box. In this example, the slant
range is from 0 to 2000 m and the azimuth range is from 0 to 1.5 ms.

\smallskip To view the graph on Microsoft Excel, follow the steps below:

\begin{itemize}
\begin{itemize}
\item  Save file in ASCII format (for this example, the LSB was 1.67572$%
\cdot 10^{-6}$ mV)

\item  Open file in Excel, the Text Import Wizard window will come up

\item  Select option `delimited' in the dialog box

\item  Click `Next' and select `Tab' and `Space' in the `Delimiters' window

\item  Click `Finish'
\end{itemize}
\end{itemize}

\FRAME{fhFU}{5.118in}{3.435in}{0pt}{\Qcb{Raw Return Window}}{\Qlb{raw.gif}}{%
raw.gif}{\special{language "Scientific Word";type
"GRAPHIC";maintain-aspect-ratio TRUE;display "USEDEF";valid_file "F";width
5.118in;height 3.435in;depth 0pt;original-width 459.9375pt;original-height
307.875pt;cropleft "0";croptop "1";cropright "1";cropbottom "0";filename
'H:/helvin/RADAR/Raw.gif';file-properties "XNPEU";}}

Figure \ref{simt1.gif}\ shows some of the values obtained from the
simulation.

\FRAME{fhFU}{3.659in}{2.7103in}{0pt}{\Qcb{Excel Worksheet for Simulation}}{%
\Qlb{simt1.gif}}{simt1.gif}{\special{language "Scientific Word";type
"GRAPHIC";maintain-aspect-ratio TRUE;display "USEDEF";valid_file "F";width
3.659in;height 2.7103in;depth 0pt;original-width 328.25pt;original-height
242.375pt;cropleft "0";croptop "1";cropright "1";cropbottom "0";filename
'H:/helvin/RADAR/simt1.gif';file-properties "XNPEU";}}

The spreadsheet will give a series of I and Q values. The number of values
will depend on the sampling frequency and the slant range used. For this
example, the sampling frequency was 20 times the Nyquist rate (0.2 MHz) and
the slant range was from 0 to 2000m. The number of values obtained for one
single pulse was 53. The magnitude, phase and slant range can be calculated
from these values as follows:

\[
\text{Magnitude}=\sqrt{\text{I}^{2}+\text{Q}^{2}}\text{.} 
\]

\[
\text{Phase}=(atan(\frac{\text{Q}}{\text{I}})-\frac{\text{180}}{\pi })\pm
n\pi \text{ where n is an integer .} 
\]

\[
\text{Slant range}=\frac{\text{c}}{\text{f}_{s}}. 
\]

Figure \ref{sim1.gif}\ displays the graph of Magnitude and Phase against
Slant Range.

\FRAME{fhFU}{4.4668in}{2.7268in}{0pt}{\Qcb{Graph of Magnitude and Phase
against Slant Range}}{\Qlb{sim1.gif}}{sim1.gif}{\special{language
"Scientific Word";type "GRAPHIC";maintain-aspect-ratio TRUE;display
"USEDEF";valid_file "F";width 4.4668in;height 2.7268in;depth
0pt;original-width 401.25pt;original-height 243.9375pt;cropleft "0";croptop
"1";cropright "1";cropbottom "0";filename
'H:/helvin/RADAR/sim1.gif';file-properties "XNPEU";}}

\subsection{\protect\smallskip Simulation Results in Matlab}

The program to view the magnitude of the pulse is given below:
\begin{verbatim}
 
load F:\filename.asc  %loads the I and Q values from the ASCII file into a
                      %matrix called `filename'
 
X = filename (:,1);   %X is a vector representing the I values of 'filename'
Y = filename (:,2);   %Y is the vector representing the Q values of
                      %`filename'
 
Z = X + i*Y;          %Z is a vector of complex numbers
B = abs(Z);           %B is a vector representing the magnitude of Z
k = length(B);        %assigns the length of the vector to a variable k
 
d=1:k;              
 
W=(299792500/8E6)*d;  %converts from number of pulses to range in meters
 
plot(B(1:k))          %plots the graph of B against W
 
title(`Graph of Magnitude against Slant Range')  %title of graph
xlabel(`Slant Range (m)')                        %labelling of x-axis
ylabel(`Magnitude (mV)')                         %labelling of y-axis
 
\end{verbatim}

The graph is shown in Figure \ref{mat.gif}.

\FRAME{fhFU}{4.4348in}{3.3935in}{0pt}{\Qcb{Graph of Magnitude against Slant
Range }}{\Qlb{mat.gif}}{mat.gif}{\special{language "Scientific Word";type
"GRAPHIC";maintain-aspect-ratio TRUE;display "USEDEF";valid_file "F";width
4.4348in;height 3.3935in;depth 0pt;original-width 398.25pt;original-height
304.125pt;cropleft "0";croptop "1";cropright "1";cropbottom "0";filename
'H:/helvin/RADAR/mat.gif';file-properties "XNPEU";}}

\smallskip

\smallskip

\subsection{\protect\smallskip Simulation Results in IDL}

The program to view the Magnitude and the Phase of the pulse is given below:
\begin{verbatim}
 
filename = 'sim11.asc'
arrlength = 53             ;assigns the number of values in the file to a
                           ;variable called arrlength
A = complexarr(arrlength)  ;creates an array of complex numbers of size 53
X = fltarr(54)             ;creates an array of floating-point numbers
line = ``                  ;creates a string called 'line'
realpart = 0.0             ;initializes the real part of the complex array
imagpart = 0.0             ;initializes the imaginary part of the complex array
 
Openr, 1, filename         ;opens the file 'filename' for reading only
 
For i = 0,(arrlength - 1) Do Begin
readf, 1, line             ;reads each line of the file filename into the
                           ;string called 'line'
reads, line, realpart, imagpart ;the components of the line is read into
                                ;variables 'realpart' and 'imagpart'
A(i) = complex(realpart, imagpart);stores the values into the complex array
 
Endfor
 
Close, 1                        ;closes the file 'filename'
 
B = abs(A)                      ;returns the magnitude
C = atan(imaginary(A), float(A));returns the phase
D = C*57.3                      ;converts the phase from radians to degrees
 
For j = 0,(arrlength) Do Begin
X(j) = (299792500/8E6)*j        ;converts the number of pulses into
                                ;range(m)and stores them into an array
Endfor
 
;Routine to plot the graphs of Magnitude and Phase against Slant Range
;In order to plot the two graphs, which have different min values and
;max values, the Oplot and Yrange procedures are used.
 
Plot,X,B,Yrange = [-120,150], $ ;plots the graph of Magnitude against
                                ;Slant Range with the range of the Y-axis
                                ;between (-120,150)
 
Title='Graph of Magnitude and Phase against Slant Range',$ ;title of the graph
Xtitle = 'Slant Range (m)'                   ;title for the X-axis
Ytitle = 'Magnitude (mV) and Phase (Degrees) ;title for the Y-axis
 
Oplot,X,D                       ;plots the graph of Phase against Slant
                                ;Range on the same axes as the previous graph
 
End
\end{verbatim}

The graph is shown in figure \ref{idl.gif} .

\FRAME{fhFU}{5.0349in}{2.5348in}{0pt}{\Qcb{Graph of Magnitude and Phase
against Slant Range}}{\Qlb{idl.gif}}{idl.gif}{\special{language "Scientific
Word";type "GRAPHIC";maintain-aspect-ratio TRUE;display "USEDEF";valid_file
"F";width 5.0349in;height 2.5348in;depth 0pt;original-width
452.4375pt;original-height 226.625pt;cropleft "0";croptop "1";cropright
"1";cropbottom "0";filename 'H:/helvin/RADAR/idl.gif';file-properties
"XNPEU";}}

\section{\protect\smallskip A stationary point target}

The purpose of this simulation is to demonstrate

\begin{itemize}
\item  the phase shift of the reflected pulse due to range delay

\item  power loss due to distance

\item  pulse compression by using matched filters
\end{itemize}

\subsection{Setup}

\smallskip This very simple example consists of a radar located at the
origin and a stationary point target on the x axis at a distance of 1
kilometer.

\smallskip The radar parameters are as follows:

\begin{itemize}
\item  \smallskip Radar Position: origin of the `Earth' coordinate system.

\item  \smallskip Pulse Type: monochrome.

\item  \smallskip Pulsewidth: T$_{P}=5000$ ns.

\item  \smallskip Pulse Repetition Frequency: 1000 Hz.

\item  \smallskip Carrier Frequency: 1 GHz.

\item  \smallskip Output Power: 1 kW.

\item  Target cross-section : 1 m$^{2}$
\end{itemize}

\smallskip \smallskip Figure \ref{1} shows the geometrical setup of the
radar and the point target.\smallskip

\FRAME{fhFU}{2.6844in}{2.2779in}{0pt}{\Qcb{ Geometry Setup}}{\Qlb{1}}{%
radar3.gif}{\special{language "Scientific Word";type
"GRAPHIC";maintain-aspect-ratio TRUE;display "ICON";valid_file "F";width
2.6844in;height 2.2779in;depth 0pt;original-width 640.625pt;original-height
542.75pt;cropleft "0";croptop "1";cropright "1";cropbottom "0";filename
'J:/SARSIM2/DOC/EXAMPLES/Radar3.gif';file-properties "XNPEU";}}

\subsection{\protect\smallskip \protect\smallskip Results}

A typical simulation window is shown in figure \ref{2}. Slow time (azimuth)
is presented on the Y-axis (increasing from bottom to top) while fast time
(shown as slant range) is displayed on the X-axis.

\FRAME{fhFU}{2.3912in}{3.4886in}{0pt}{\Qcb{Magnitude against Slant Range for
Raw Return}}{\Qlb{2}}{statargr.gif}{\special{language "Scientific Word";type
"GRAPHIC";maintain-aspect-ratio TRUE;display "ICON";valid_file "F";width
2.3912in;height 3.4886in;depth 0pt;original-width 380.1875pt;original-height
556.3125pt;cropleft "0";croptop "1";cropright "1";cropbottom "0";filename
'H:/helvin/RADAR/statargr.gif';file-properties "XNPEU";}}

Several facts can be observed from Figure \ref{2}. The target is stationary,
positioned at 1 km as specified. The pulse extends over 750 m in range. The
actual pulse length is twice that ($c\cdot T_{P}=1500$ m), the discrepancy
is due to the fact that range and not range delay is shown on the X-axis. In
other words the time scale has been divided by a factor of two, as the pulse
has to travel back and forth. The pulse repetition interval (PRI) is 1 ms.

After saving the data the phase and magnitude of the pulse can be
determined. Usually data is sampled with 8 bit or 12 bit accuracy, for this
example 8 bits will be used. The least significant bit value is chosen such
that the signal makes optimum use of the dynamic range without saturation
occurring. (In reality there will always be some saturation occurring due to
noise.)

For the given data the following values were obtained :

Least significant bit value = $1.6757\cdot 10^{-6}$ mV

I = -24 = (-24-127)$\cdot 1.6757\cdot 10^{-6}$ mV = -253 $\ \mu $V (in-phase
component)

Q = -121 = (-121-127)$\cdot 1.6757\cdot 10^{-6}$ mV = -415.6 $\mu $V
(quadrature component)

If the data is saved in binary format an offset of $2^{n-1}-1$ is added,
where n specifies the number of bits used.

\subsection{Analysis}

\subsubsection{Sampling spacing}

In this specific simulation the data has been sampled at 20 times the
Nyquist rate.

$f_{s}=20\cdot 2\cdot \frac{1}{T_{P}}=8$ MHz

For the given range window of 2 km shown in figure \ref{2}, 53 samples were
taken. Each sample corresponds to $\frac{c}{2f_{s}}$ meter in range where $%
c=299792500$ $\frac{\text{m}}{\text{s}}$.

\subsubsection{\protect\smallskip Power loss}

\smallskip Although the return pulse is of the same shape as the radiated
one, the magnitude is significantly less than the radiated pulse. The
received power can be estimated by the following formula :

\[
\text{P}_{received}=\frac{\text{P}_{transmitted}\cdot G^{2}\cdot \lambda
^{2}\cdot \sigma }{(4\cdot \pi )^{3}\cdot R^{4}} 
\]

$\sigma =$ target cross-section

$\lambda =$ wavelength of pulse

$R=$ distance to target

$G=$ Antenna gain

For the given scenario the received power would be :

\[
\text{P}_{received}=\frac{\text{P}_{transmitted}\cdot G^{2}\cdot \lambda
^{2}\cdot \sigma }{(4\cdot \pi )^{3}\cdot R^{4}}=\frac{1000\cdot 1\cdot
0.2998^{2}\cdot 1}{(4\cdot \pi )^{3}\cdot 1000^{4}}=45.293269\cdot 10^{-15}%
\text{ W} 
\]

which gives a corresponding amplitude (assuming the ubiquitous 1 $\Omega $
resistor) of :

\[
\text{Magnitude}=\sqrt{\text{P}_{received}}=2.1282\cdot 10^{-4}\text{ mV} 
\]
mV

The simulated values give a magnitude of :

$\sqrt{I^{2}+Q^{2}}=2.128\cdot 10^{-4}$ mV

\smallskip

\subsubsection{\protect\smallskip \protect\smallskip \protect\smallskip %
Phase shift}

The phase of the return pulse is determined by the wavelength and the
distance to the radar and is given by :

\[
\text{Phase}=2\cdot \pi \cdot (-2\cdot d\cdot \frac{f}{c}) 
\]

where c is the velocity of light and\ d is the radar-target distance. As the
phase wraps around every 2$\pi $ radians, only the remainder can be
determined. For a point target at a distance of 1000 m :

\begin{equation}
\text{Phase}=\func{mod}(2\cdot \pi \cdot (-2\cdot d\cdot \frac{f}{c}),2\cdot
\pi )=\func{mod}(2\cdot \pi \cdot (-2\cdot 1000\cdot \frac{10^{9}}{299792500}%
),2\cdot \pi )=-1.7653=101.15\deg
\end{equation}

\ \ \ \ \ \ \ \ The simulation values give the same value :

Phase$=-\frac{\pi }{2}-\arctan (\frac{-24}{-121})=-101.22\deg .$

The slight discrepancy is due to the 8 bit quantization of the
samples.\smallskip

\subsubsection{Range Compression}

\smallskip In pulse compression, the return pulse from the raw return is
compressed.\smallskip This is achieved by using a matched filter. The
concept of range compression will be explained in more detail in the
following section.

\smallskip

\smallskip The choice of a radar waveform will affect radar performance in
terms of range, Doppler, and angle measurements, as well as the system's
detection performance. For example, a short monochromatic pulse of
timelength t$_{p}$ will allow resolution of

\[
\text{R}_{s}=\frac{c\cdot t_{p}}{2}. 
\]
\smallskip

Therefore, targets that are closer in distance than R$_{s}$ cannot be easily
``resolved'' or separated, because the returns will overlap. The important
parameter for range resolution is bandwidth; thus, a monochrome pulse has an
effective bandwidth B$_{p}=\frac{1}{t_{p}}$. The effective bandwidth for a
monochrome pulse waveform is the bandwidth of the pulse spectrum. By making
a pulse shorter, or the effective bandwidth larger, range resolution is
improved. However, the amount of energy contained in the pulse diminishes as
the pulse shortens.

Improved detection is achieved by proper filtering or processing of the
received signal. The use of a matched filter optimizes performance in the
presence of white,Gaussian noise. The return pulse is convolved with a
conjugate replica of itself. Since the pulse is rectangular in shape, a
convolution with itself will give a triangular pulse.

As can be seen from figure \ref{3} the resultant pulse is triangular in
shape. Since range-compression is computational intensive the sampling rate
can be set. Using low values for the sampling rate (less than the Nyquist
frequency) should be avoided due to aliasing. In this example, the sampling
rate used is hundred (100) times the Nyquist frequency. It is to be noted
that neither the magnitude nor the phase changes when the return pulse is
range compressed.\smallskip \FRAME{fhFU}{2.1447in}{3.103in}{0pt}{\Qcb{%
Magnitude against Slant Range for Range Compression}}{\Qlb{3}}{statargc.gif}{%
\special{language "Scientific Word";type "GRAPHIC";maintain-aspect-ratio
TRUE;display "ICON";valid_file "F";width 2.1447in;height 3.103in;depth
0pt;original-width 383.1875pt;original-height 556.3125pt;cropleft
"0";croptop "1";cropright "1";cropbottom "0";filename
'H:/helvin/RADAR/Statargc.gif';file-properties "XNPEU";}}

The importance of pulse compression will be demonstrated in more detail in
the next example.

\section{\protect\smallskip \protect\smallskip A moving point target}

The previous simulation demonstrated fixed point targets. For this
simulation the target will move relative to the radar. The purpose is to
demonstrate:

\begin{itemize}
\item  moving point targets

\item  chirp modulation

\item  matched filters

\item  range resolution

\item  Doppler frequency shift

\item  the concept of moving platforms
\end{itemize}

A platform is a convenient way of grouping targets, which are not moving
relative to each other, together. Their position and velocity can then be
set by just defining the trajectory of that specific platform.

\subsection{Setup}

\smallskip \smallskip The radar parameters are as follows:

\begin{itemize}
\begin{itemize}
\item  Radar position: origin of the `Earth' coordinate system.

\item  \smallskip Pulse Type: chirp.

\item  Bandwidth: 0.1 GHz.

\item  \smallskip Pulsewidth, T$_{p}$: 2500 ns.

\item  Pulse Repetition Frequency: 1 kHz.

\item  Carrier Frequency: 0.5 GHz.

\item  Output power: 1 kW.

\item  Target Cross-section:1 m$^{2}$.
\end{itemize}
\end{itemize}

\smallskip There are two targets that are 8 meters apart. They have been
grouped on a platform and the parameters for the platform are:

\begin{itemize}
\begin{itemize}
\item  \smallskip \smallskip X = 300 m and moving at a speed of 50 ms$^{-1}$
in the x-direction,

\item  Y = 0,

\item  Z = 0.
\end{itemize}
\end{itemize}

The geometrical setup is given in figure \ref{4}.

\FRAME{fhFU}{2.0072in}{2.6342in}{0pt}{\Qcb{Geometry Setup}}{\Qlb{4}}{%
movetarg.gif}{\special{language "Scientific Word";type
"GRAPHIC";maintain-aspect-ratio TRUE;display "ICON";valid_file "F";width
2.0072in;height 2.6342in;depth 0pt;original-width 318.4375pt;original-height
419.3125pt;cropleft "0";croptop "1";cropright "1";cropbottom "0";filename
'J:/SARSIM2/DOC/EXAMPLES/Movetarg.gif';file-properties "XNPEU";}}

\subsection{Results}

\subsubsection{Raw Return}

\smallskip The pulse type used in this example is Chirp pulse. For chirp
waveform, the frequency, f$_{0}$ is not kept constant throughout the pulse,
but is linearly changed from f$_{0}+\Delta $f. $\Delta $f can be positive
(up chirp) or negative (down chirp). The result is a waveform which has a
bandwidth that is independent of the pulse length T$_{p}$. Thus , a pulse
with large T$_{p}$ and large B can be construted. The reason for using a
chirp waveform is, to have a high energy pulse which increases the range
resolution capability of the radar.The transmit waveform is generated with a
frequency versus time domain waveform as shown in the figure \ref{5}.

\FRAME{fhFU}{3.0329in}{2.1508in}{0pt}{\Qcb{Frequency against time domain
waveform}}{\Qlb{5}}{chirp.gif}{\special{language "Scientific Word";type
"GRAPHIC";maintain-aspect-ratio TRUE;display "ICON";valid_file "F";width
3.0329in;height 2.1508in;depth 0pt;original-width 292.875pt;original-height
207pt;cropleft "0";croptop "1";cropright "1";cropbottom "0";filename
'H:/helvin/RADAR/Chirp.gif';file-properties "XNPEU";}}

Figure \ref{6} shows the raw return for the simulation (only one pulse is
shown here for clarity). Due to the fact that there are two targets that are
close together, it is very difficult to separate the two targets from each
other. The received pulse consists of the return pulse from the two targets
interfering with each other.

\FRAME{fhFU}{3.7654in}{2.3341in}{0pt}{\Qcb{Simulation Window for Real Part
of Raw Return}}{\Qlb{6}}{movr.gif}{\special{language "Scientific Word";type
"GRAPHIC";maintain-aspect-ratio TRUE;display "ICON";valid_file "F";width
3.7654in;height 2.3341in;depth 0pt;original-width 901.125pt;original-height
556.3125pt;cropleft "0";croptop "1";cropright "1";cropbottom "0";filename
'H:/helvin/RADAR/movr.gif';file-properties "XNPEU";}}

\subsubsection{Range Compression}

\smallskip In order to be able to distinguish between the two targets, range
compression is used. This is shown in Figure \ref{7}. From the figure, it
can be seen that the first target is at 300 m and the second one is at 308 m.

\FRAME{fhFU}{3.0978in}{2.7112in}{0pt}{\Qcb{Magnitude of Signal after Range
Compression}}{\Qlb{7}}{movc.gif}{\special{language "Scientific Word";type
"GRAPHIC";maintain-aspect-ratio TRUE;display "ICON";valid_file "F";width
3.0978in;height 2.7112in;depth 0pt;original-width 634.625pt;original-height
554.8125pt;cropleft "0";croptop "1";cropright "1";cropbottom "0";filename
'H:/helvin/RADAR/movc.gif';file-properties "XNPEU";}}

The theoretical resolution of the radar is determined by the using the
following equation:

$\Delta $R$_{s}=\frac{c}{2\cdot B}$

where B = bandwidth of the chirp pulse.

For the given bandwidth of 100 MHz, the resolution will be 1.5
meters.\smallskip

\subsubsection{Phase Shift}

\smallskip The first return of the first 9 pulses (the first pulse sent at
time t=0 seconds) were sampled (after range compression) and the
corresponding phase values are shown in Figure \ref{targ2}. As the platform
is moving into positive x-direction, this will introduce a phase shift of
the return array for every pulse. The PRF = 1 kHz, therefore the target will
move by 0.05 m/pulse. This corresponds to a phase shift of $2\cdot \pi (%
\frac{-2\cdot d}{\lambda })=-1.0472$ radians as verified in Figure \ref
{targ2}. Point targets 1 and 2 are positioned at 300 m and 308 m
respectively at time t=0 seconds.

\FRAME{fhFU}{4.0231in}{2.8963in}{0pt}{\Qcb{Phase shift}}{\Qlb{targ2}}{%
targ2.eps}{\special{language "Scientific Word";type
"GRAPHIC";maintain-aspect-ratio TRUE;display "ICON";valid_file "F";width
4.0231in;height 2.8963in;depth 0pt;original-width 340.25pt;original-height
243.9375pt;cropleft "0";croptop "1";cropright "1";cropbottom "0";filename
'J:/SARSIM2/DOC/EXAMPLES/Targ2.eps';file-properties "XNPEU";}}

\section{Search Radar}

This example consists of a rotating antenna beam with a certain antenna gain
pattern an a single point target. A measure of the ability of an antenna to
concentrate energy in a particular direction is called the gain. Two
different, but related definitions of antenna gain are the directive gain
and the power gain. The directive gain is descriptive of the antenna pattern.

The directive gain of a transmitting antenna is defined as the
ratio\smallskip 
\[
G_{D}=\frac{\text{maximum radiation intensity}}{\text{average radiation
intensity}} 
\]

\smallskip The radiation intensity is the power per unit solid angle in
direction ($\eta ,\epsilon $) and is denoted by P($\eta ,\epsilon $). The
two-angle coordinates commonly used with ground-based antennas are azimuth
angle $\eta $ and elevation angle $\epsilon $.

A simple approximation to typical antenna gain patterns is the $\frac{\sin
(x)}{x}$ shape. This can be analytically calculated by considering the
current distributions across a circular aperture. The power radiation
pattern is \ a ($\frac{\text{sin(x)}}{\text{x}}$)$^{2}$ pattern. Figure \ref
{gain} shows an example for a beamwidth of 10 degrees.

\FRAME{fhFU}{3.4221in}{2.3454in}{0pt}{\Qcb{Magnitude gain pattern for a
beamwidth of 10 degrees}}{\Qlb{gain}}{ejfck000.wmf}{\special{language
"Scientific Word";type "GRAPHIC";maintain-aspect-ratio TRUE;display
"ICON";valid_file "F";width 3.4221in;height 2.3454in;depth
0pt;original-width 231.5625pt;original-height 158.25pt;cropleft "0";croptop
"1";cropright "1";cropbottom "0";filename
'J:/SARSIM2/DOC/EXAMPLES/Ejfck000.wmf';file-properties "XNPEU";}}

The half-power points would be at an offset of 5 degrees of the beam center.
The amplitude gain is calculated by :

\[
\text{Gain}=\frac{\text{sin}(x)}{x},\text{ }x=0.88\cdot \pi \cdot \frac{%
\text{sin}(\theta )}{\text{Beamwidth}} 
\]

\subsection{Setup}

\smallskip This example consists of a search radar and a stationary point
target.\smallskip The simulation parameters for the radar are as follows:

\begin{itemize}
\item  Monochrome pulse modulation, pulsewidth 5000 ns

\item  \smallskip Pulse repetition frequency: 400 Hz

\item  \smallskip Carrier frequency: 1 GHz

\item  \smallskip Output power: 1 kW

\item  \smallskip \emph{Sin(x)/x} transmitter antenna gain, elevation beam
width 30 degrees,\ azimuth beam width 1 degrees

\item  \smallskip Rotation rate 24 deg/s, beam elevation angle 30 degrees

\item  Target offset : X = 1000 m, \smallskip Y = 1000 m, Z = 816 m
(distance = 1632.74 m)
\end{itemize}

\FRAME{fhFU}{2.6705in}{2.5365in}{0pt}{\Qcb{Geometry setup}}{\Qlb{geo}}{%
rotrad.gif}{\special{language "Scientific Word";type
"GRAPHIC";maintain-aspect-ratio TRUE;display "ICON";valid_file "F";width
2.6705in;height 2.5365in;depth 0pt;original-width 478.0625pt;original-height
453.9375pt;cropleft "0";croptop "1";cropright "1";cropbottom "0";filename
'J:/SARSIM2/DOC/EXAMPLES/Rotrad.gif';file-properties "XNPEU";}}

\subsection{\protect\smallskip Results}

The transmitter antenna gain pattern is a simple Sin(x)/x, the power
radiation pattern of the return pulse is a [sin(x)/x]$^{2}$. As can be seen
from Figure \ref{8}, the return pulse has a main lobe and side lobes of
decreasing magnitude.

\smallskip \FRAME{fhFU}{5.5677in}{3.4229in}{0pt}{\Qcb{Graph of Magnitude
against Sample Number for Raw Return}}{\Qlb{8}}{searchr.gif}{%
\special{language "Scientific Word";type "GRAPHIC";maintain-aspect-ratio
TRUE;display "ICON";valid_file "F";width 5.5677in;height 3.4229in;depth
0pt;original-width 455.4375pt;original-height 279.3125pt;cropleft
"0";croptop "1";cropright "1";cropbottom "0";filename
'H:/helvin/RADAR/Searchr.gif';file-properties "XNPEU";}}

\smallskip

\subsubsection{Power Loss}

\smallskip The transmitted power is 1 kW and the received power at the beam
center is calculated as follows:\smallskip 
\[
\text{P}_{received}=\frac{\text{P}_{transmitted}\cdot G^{2}\cdot \lambda
^{2}\cdot \sigma }{(4\cdot \pi )^{3}\cdot R^{4}}=\frac{1000\cdot 1\cdot
0.2998^{2}\cdot 1}{(4\cdot \pi )^{3}\cdot (1632.745)^{4}}=6.37324\cdot
10^{-15}\text{ W} 
\]

\smallskip 
\[
\text{Magnitude}=\sqrt{\text{P}_{received}}=7.9833\cdot 10^{-5}\text{ mV} 
\]

\smallskip

The value obtained from the simulation is $7.97376\cdot 10^{-5}$ mV

\subsubsection{\protect\smallskip Phase Calculation}

\smallskip \smallskip The (x,y,z) coordinates for the target are
(1000,1000,816).

\smallskip The distance of the target from the radar is $d=1632.744928$ m$.$

\smallskip $f=1$ GHz

\smallskip $c=299792500$ ms$^{-1}$

\smallskip

\[
\text{Phase}=\func{mod}(2\cdot \pi \cdot (-2\cdot d\cdot \frac{f}{c}),2\cdot
\pi )\approx -\pi 
\]

\smallskip

The phase is $-180$ degrees and from the simulation values, it can be seen
that the Q values are all zero and the I values are negative.

\subsection{\protect\smallskip \protect\smallskip Range Compression}

\smallskip The range compression result is shown in Figure \ref{9}. This was
obtained by convolving the raw return pulse with a conjugate replica of
itself.

\subsubsection{Resolution}

The theoretical resolution of the radar is

$\Delta $R$_{s}=\frac{c}{2\cdot B}=\frac{299792500\cdot T_{p}}{2}=$ $749.5$

\FRAME{fhFU}{5.4699in}{3.3451in}{0pt}{\Qcb{Magnitude against Sample Number
for Range Compression.}}{\Qlb{9}}{searchco.gif}{\special{language
"Scientific Word";type "GRAPHIC";maintain-aspect-ratio TRUE;display
"ICON";valid_file "F";width 5.4699in;height 3.3451in;depth
0pt;original-width 457.6875pt;original-height 279.3125pt;cropleft
"0";croptop "1";cropright "1";cropbottom "0";filename
'H:/helvin/RADAR/Searchco.gif';file-properties "XNPEU";}}

The difference between the main lobe and the first side lobe levels is 13.5
dB.

With filtering, the side lobes can be reduced at the expense of resolution.

\end{document}
