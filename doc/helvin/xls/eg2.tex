
\documentclass{sebase}
%%%%%%%%%%%%%%%%%%%%%%%%%%%%%%%%%%%%%%%%%%%%%%%%%%%%%%%%%%%%%%%%%%%%%%%%%%%%%%%%%%%%%%%%%%%%%%%%%%%%%%%%%%%%%%%%%%%%%%%%%%%%
\usepackage{SEART}

%TCIDATA{TCIstyle=article/art4.lat,SEART,SEART}

%TCIDATA{Created=Wed Oct 15 12:20:00 1997}
%TCIDATA{LastRevised=Tue Oct 28 11:08:54 1997}
%TCIDATA{Language=American English}

\input{tcilatex}
\begin{document}

\SetTitle{Tutorial Manual for SARSIM II}
\SetAuthor{H. R. Gunputh}
\Setdate{October 20, 1997}

\section{\protect\smallskip }

\subsection{Example 2}

The next example consists of a radar sitting at the origin of the 'Earth'
coordinate system and a point target moving at a speed of $200ms^{-1}$on the
x axis.

\smallskip

The simulation parameters for the radar are the same as example 1 except for
the Pulse Repetition Frequency which has been changed to 1Hz

\smallskip

\smallskip The simulation parameters for the target are as follows:

\smallskip

Position:

\smallskip

X = 1000 m and moving along the X-axis at a speed of $200$ $ms^{-1}.$

\smallskip 

Y = 0 m,

\smallskip

Z = 0 m.

\smallskip

Figure 2a

\smallskip

\smallskip

\smallskip

\smallskip

\smallskip

\smallskip

\smallskip

\smallskip

\smallskip

\smallskip

\smallskip

\smallskip

\smallskip

\smallskip

\smallskip

\smallskip

\smallskip 

\subsubsection{Raw Return Results}

\smallskip

\smallskip The pulse repetition frequency (PRF) is 1 Hz, i.e. the radar is
sending a square pulse every 1 Second. The shape of the return pulse is also
square but this time, the slant range will be different for each since the
target is moving away from the radar. In this example. The sampling rate is
still twenty (20) times the Nyquist Frequency.

\smallskip

The figure below displays the I, Q values obtained from the simulation, the
Magnitude of the pulse in mV against Slant Range.

\smallskip

\smallskip Figure 2b

\smallskip

Graph for Raw Return

\smallskip

\smallskip

\smallskip

\smallskip

\smallskip

\smallskip

\smallskip

\smallskip

\smallskip

\paragraph{\protect\smallskip Range Calculation}

\smallskip

The range from the radar to the target can be read off from the slant range
axis, and in this case it is 1000m which corresponds to the value used for
the X coordinate.\smallskip

\smallskip The range can be calculated from the simulation values as follows:

$R_{win}=\frac{c}{2f_{s}}\cdot N$

where R$_{\text{win}}$ is the size of the slant range window,

\ \ \ \ \ \ \ \ \ c is the velocity of light,

\ \ \ \ \ \ \ \ \ f$_{\text{s}}$ is the sampling frequency.

\ \ \ \ \ \ \ \ \ N is the number of simulation values.

In this case, using the values

$N$ $=53$, $c=$ $299792500$ $ms^{-1}$, $f_{s}=2\cdot 20\cdot 0.0002$ $GHz$

\smallskip

$R_{win}=$ $\frac{299792500}{2\cdot 4\cdot 10^{6}}$ $\cdot 53$

\ \ \ \ \ \ $\ =$ $1986.125\ m$

\smallskip

\paragraph{Power Loss Calculation}

\smallskip

Although the return pulse is of the same shape as the radiated one, the
magnitude is significantly less than the radiated pulse.

\smallskip Some of the electromagnetic energy intercepted by the target is
absorbed as heat and the rest is scattered. Portions of energy scattered in
the direction of the receiving antenna are gathered by the antenna and
subsequently processed by target detection circuitry. Hence the amount of
energy received by the antenna is dependent on the \textit{radar target
cross section (}$\sigma $).

\smallskip

The transmitted power is 1 kW and the received power is calculated from the
following equation:

$\smallskip $

$P_{received}=\frac{P_{transmitted}\cdot G^{2}\cdot \lambda ^{2}\cdot \sigma 
}{(4\cdot \pi )^{3}\cdot R^{4}}=\frac{1000\cdot 1\cdot 0.2998^{2}\cdot 1}{%
(4\cdot \pi )^{3}\cdot 1000^{4}}=45.293269\cdot 10^{-15}W$

\smallskip

\smallskip and the amplitude as shown:

$\smallskip $

$Magnitude=\sqrt{P_{received}}=2.1282\cdot 10^{-4}mV$

\smallskip

\paragraph{\protect\smallskip Phase Calculation}

The figure below shows the Phase in Degrees against Slant Range. The phase
of the return pulse is calculated as follows:

\smallskip

$Phase=2\cdot \pi \cdot (-2\cdot d\cdot \frac{f}{c})$

\ \ \ \ \ \ \ \ \ \ $=2\cdot \pi \cdot (-2\cdot d\cdot \frac{1}{\lambda })$\ 

\smallskip where c is the velocity of light,

\ \ \ \ \ \ \ \ \ d is the radar-target distance.

For this example, the value read from the graph is -78.78 degrees and the
calculated value is

\smallskip

\subsubsection{Range Compression Results}

\smallskip

In range compression the return pulse from the raw return gets
compressed.\smallskip This is achieved by using a matched filter. The return
pulse is convolved with a conjugate replica of itself. Since the pulse is
rectangular in shape, a convolution with itself will give a triangular pulse.

As can be seen from figure \#3 below the resultant pulse is triangular in
shape. Since range-compression is computational intensive the sampling rate
can be set. Using low values for the sampling rate (less than three times
the Nyquist frequency) can result in unfamiliar graphs. In this example, the
sampling rate used is fifty (50) times the Nyquist frequency.

\smallskip

\smallskip

Figure \#3

\smallskip

\smallskip Graph for Range Compression.

\smallskip

\smallskip

\smallskip

\smallskip

\smallskip

\end{document}
