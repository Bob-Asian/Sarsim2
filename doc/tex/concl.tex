%TCIDATA{LaTeXparent=0,0,Thesis.tex}
                      

\chapter{Conclusions\label{chapter:concl}}

The radar simulator described in this dissertation has proven to be
extremely useful in providing simulated data used for SAR processing
applications, for stepped-frequency processing applications and also for
aircraft and ship recognition analysis. The graphical user-interface makes
it easy to use, since one can see and analyse the returned waveforms before
writing them to disk. Furthermore, due to efficient programming, it is
extremely fast, since only the appropriate computations (corresponding to
the current screen resolution) are executed.

\section{Future work}

As with any software developed, there is always scope for improvement. The
following items are suggestions for future work:

\begin{itemize}
\item  Implementation of surfaces. At the moment only point target
simulations are possible, however for proper SAR and Interferometric SAR
applications it would be of great benefit to implement proper surface
simulations.

\item  Bandlimiting of returned (and transmitted) waveforms to reduce
aliasing. Theoretical radar pulses are never truly bandlimited, due to the
sharp cutoffs at the signal boundaries. This can lead to aliasing problems,
even if the returned waveform is sampled well above the Nyquist rate.

\item   Looking at a JAVA implementation, in order to be able to distribute
a large simulation over a number of machines, and to make the simulator
platform independent, i.e.~the graphical interface in particular.

\item  Including satellite ephemeris platform trajectories, to facilitate
spaceborne SAR simulation.

\item  Separate motion scenario generator, to provide motion files for radar
and targets in complex airborne radar environments.
\end{itemize}
