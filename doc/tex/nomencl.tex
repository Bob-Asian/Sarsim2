%TCIDATA{LaTeXparent=0,0,thesis.tex}
                      

\chapter*{Nomenclature}

\addcontentsline{toc}{chapter}{Nomenclature}

\textbf{Azimuth---}Angle in a horizontal plane, relative to a fixed
reference, usually north or the longitudinal reference axis of the aircraft
or satellite.

\textbf{Beamwidth---}The angular width of a slice through the mainlobe of
the radiation pattern of an antenna in the horizontal, vertical or other
plane.

\textbf{Burst---}Set of all frequencies required to produce a synthetic
range profile.

\textbf{C-band---}The frequency range centered around 5\thinspace GHz.

\textbf{Chirp---}A pulse modulation method used for pulse compression, also
called \textit{linear frequency modulation}. The frequency of each pulse is
increased or decreased at a constant rate throughout the length of the pulse.

\textbf{Coherence---}A continuity or consistency in the phases of successive
radar pulses.

\textbf{Corner reflector---}A radar reflector that reflects nearly all of
the radio frequency energy it intercepts back in the direction of the radar
which is illuminating it.

\textbf{Dilute---}If individual scatterers on a target can be resolved, the
target features are said to be \textit{dilute}.

\textbf{Doppler frequency---}A shift in the radio frequency of the return
from a target or other object as a result of the object's radial motion
relative to the radar.

``\textbf{Earth'' platform}---Stationary, unshifted, non-rotating coordinate
system, identical to the visible axes on the main screen.

\textbf{Encounter---}Set of all profiles while target in sight, acquired
over a number of scans.

\textbf{Isotropic---}Non-directional.

\textbf{Ku-band---}The frequency range centred around 14\thinspace GHz.

\textbf{LSB}---Least significant bit.

\textbf{MTI---}Moving target indication.

\textbf{Nadir---}The region directly below the satellite position.

\textbf{NCTR---}Non-cooperative target recognition.

\textbf{NSCAT---}NASA Scatterometer.

\textbf{OWL}---Object Windows Library.

\textbf{Platform}---A user-defined coordinate system which can move
independently on any path seen relative to the ``Earth'' coordinate system.
All point targets or radars defined on that platform will be stationary as
seen from that coordinate system.

\textbf{Point Target---}Infinitely small point which reflects
electromagnetic energy. The amount of reflection depends on its surface area
and its directional vector.

\textbf{PRF---}Pulse repetition frequency.

\textbf{PRI}---Pulse repetition interval.

\textbf{Profile---}A single synthetic range profile of a target.

\textbf{RAD}---Rapid Application Development.

\textbf{Radar}---Actual energy-transmitting device. Can be positioned onto
any platform, however the position is always at the origin of that platform.

\textbf{Range---}The radial distance from a radar to a target.

\textbf{RCS}---Radar cross section.

\textbf{RRSG}---Radar Remote Sensing Group.

\textbf{Scan---}Set of pulses received during illumination time.

\textbf{Specular---}Highly directive, i.e.~the power returned from a
specular reflector depends very much on the direction of illumination.

\textbf{SRP---}Synthetic range profile.

\textbf{Swath---}The area on earth covered by the antenna signal.

\textbf{Synthetic Aperture Radar (SAR)---}A signal-processing technique for
improving the azimuth resolution beyond the beamwidth of the physical
antenna actually used in the radar system. This is done by synthesising the
equivalent of a very long sidelooking array antenna.

\textbf{UCT---}University of Cape Town.
